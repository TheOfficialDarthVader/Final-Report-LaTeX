\documentclass[../Interim_Report_Master]{subfiles}
\begin{document}
\hypertarget{op_imp}{\section{OpenCL Implementation}\label{op_imp}}
The initial testing of numerical methods and the model were performed in Python as this language allows for prototype code to be quickly produced due to its simplicity and feature rich libraries such as NumPy. Python also features capability for creating GPU code through the pyopencl library \cite{pyopencl}. However, the existing code uses C and OpenCL to build executable files for running simulations. 

\subsection{Additions made to the Code}
\subsubsection{Source Code Structure}
For the sake of clarity the existing source code structure is shown in Figure \ref{demorange_struct} and can be found in Section \ref{prog_strut}. The new source code structure with DEM functionality removed is shown in Figure \ref{hmorange_struct}. 
\begin{figure}
	\centering
	\includegraphics*[width=0.32\textwidth, trim=0 0 0 0, clip]{./Diagrams/HMOranges_Structure/HMOranges_Structure.pdf}
		\caption{File structure of the HMOranges codebase.}
	\label{hmorange_struct}
\end{figure}

\subsubsection{Variables}
As already mentioned the DEM code had to be stripped out. Once this was completed and results from the code verified the heat and mass transfer methods could be added. This involved adding the following set of new variables to the particle structure:
\begin{multicols}{3}
\begin{itemize}
	\item T\_d;
	\item W\_V;
	\item T\_B;
	\item L\_V;
	\item C\_L;
	\item P\_atm;
	\item R\_bar;
	\item R;
	\item W\_G;
	\item theta\_1;
	\item theta\_2;
	\item Y\_G;
	\item Pr\_G;
	\item Sc\_G;
	\item f\_2;
	\item P\_G;
	\item T\_G;
	\item H\_deltaT;
	\item m\_d;
\end{itemize} 
\end{multicols}


In addition to this code was added to the function that initialises the particles so the extra required variables are set to the required initial conditions. Finally, the equations to solve the heat and mass transfer were added to the iterate particle kernel. This included adding a stop condition for updating the particle properties. Namely: position, speed, temperature and mass. When the current particle mass reaches $1\%$ of its starting mass, all properties for the particle are set to zero. At each further timestep the value of the particle properties is logged as zero. This makes later data analysis easier as particles are identified by a row of data in the text file. So the number of rows has to remain the same. This is especially important for utilising Paraview which identifies particles by row.  

\subsubsection{Heat and Mass Transfer Model}

\subsection{Verification}
Similar to the Python code the OpenCl code must undergo verification tests to ensure the equations are being solved correctly.

\subsubsection{Uncoupled Heat Transfer}

\subsubsection{Uncoupled Mass Transfer}

\subsubsection{Coupled Heat and Mass Transfer}
\end{document}